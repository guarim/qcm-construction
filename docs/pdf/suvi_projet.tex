\documentclass[11pt,a4paper]{article}

\usepackage{Act}
\usepackage{listings}

\begin{document}
\input{\detokenize{/home/fenarius/Travail/Cours/Commun/latex/Macros.tex}}

\fancyhead[L]{\Term} \fancyhead[R]{Lycée Hintermann-Afféjee}
\fancyfoot[C]{}
\begin{center}
\framebox{\makebox[0.986\textwidth][c]{\Large \bf Suivi du projet {\sc gomoku}}} \\
\end{center}

\begin{tabularx}{0.47\textwidth}{|p{1.7cm}X|}
    \hline
    Nom    & \dotfill \\
    Prénom & \dotfill \\
    Classe & \dotfill \\
    \hline
\end{tabularx} \hfill
\begin{tabularx}{0.47\textwidth}{|p{1.7cm}X|}
    \hline
    Nom    & \dotfill \\
    Prénom & \dotfill \\
    Classe & \dotfill \\
    \hline
\end{tabularx} 

\renewcommand{\arraystretch}{1.2}
\begin{tabularx}{\textwidth}{|X|}
    \hline
    \ding{43} {\bf Etape \ding{182} :} le \dots/\dots/\dots \hfill \ding{113}\;Réussite \ \ding{113}\;Réussite partielle \ \ding{113}\;Echec\\
    Commentaires (difficultés, pistes de recherche, qualité du code, tests, organisation, rôles, \dots ):\\
    \ \dotfill \\
    \ \dotfill \\
    \ \dotfill \\
    \ \dotfill \\
    \ \dotfill \\
    \hline
\end{tabularx}
\renewcommand{\arraystretch}{1.2}
\begin{tabularx}{\textwidth}{|X|}
    \hline
    \ding{43} {\bf Etape 2 :} le \dots/\dots/\dots \hfill \ding{113}\;Réussite \ \ding{113}\;Réussite partielle \ \ding{113}\;Echec\\
    Commentaires (difficultés, pistes de recherche, qualité du code, tests, organisation, rôles, \dots ):\\
    \ \dotfill \\
    \ \dotfill \\
    \ \dotfill \\
    \ \dotfill \\
    \ \dotfill \\
    \hline
\end{tabularx}
\renewcommand{\arraystretch}{1.2}
\begin{tabularx}{\textwidth}{|X|}
    \hline
    \ding{43} {\bf Etape 3 :} le \dots/\dots/\dots \hfill \ding{113}\;Réussite \ \ding{113}\;Réussite partielle \ \ding{113}\;Echec\\
    Commentaires (difficultés, pistes de recherche, qualité du code, tests, organisation, rôles, \dots ):\\
    \ \dotfill \\
    \ \dotfill \\
    \ \dotfill \\
    \ \dotfill \\
    \ \dotfill \\
    \hline
\end{tabularx}
\renewcommand{\arraystretch}{1.2}
\begin{tabularx}{\textwidth}{|X|}
    \hline
    \ding{43} {\bf Etape 4 :} le \dots/\dots/\dots \hfill \ding{113}\;Réussite \ \ding{113}\;Réussite partielle \ \ding{113}\;Echec\\
    Commentaires (difficultés, pistes de recherche, qualité du code, tests, organisation, rôles, \dots ):\\
    \ \dotfill \\
    \ \dotfill \\
    \ \dotfill \\
    \ \dotfill \\
    \ \dotfill \\
    \hline
\end{tabularx}
\renewcommand{\arraystretch}{1.2}
\begin{tabularx}{\textwidth}{|X|}
    \hline
    \ding{43} {\bf Etape 5 :} le \dots/\dots/\dots \hfill \ding{113}\;Réussite \ \ding{113}\;Réussite partielle \ \ding{113}\;Echec\\
    Commentaires (difficultés, pistes de recherche, qualité du code, tests, organisation, rôles, \dots ):\\
    \ \dotfill \\
    \ \dotfill \\
    \ \dotfill \\
    \ \dotfill \\
    \ \dotfill \\
    \hline
\end{tabularx}
\renewcommand{\arraystretch}{1.2}
\begin{tabularx}{\textwidth}{|X|}
    \hline
    \ding{52} {\bf Bilan et note :} le \dots/\dots/\dots \\
    \ \dotfill \\
    \ \dotfill \\
    \ \dotfill \\
    \hline
\end{tabularx}

\end{document}