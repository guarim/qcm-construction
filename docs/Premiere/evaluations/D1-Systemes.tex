\documentclass[11pt,a4paper]{article}

\usepackage{Act}
\usepackage{listings}

\begin{document}
\input{\detokenize{/home/fenarius/Travail/Cours/Commun/latex/Macros.tex}}

\DNSI{Systèmes d'exploitation}{\Pre}\vspace{0.2cm}

\bashmode



\Exo{Quelques commandes}{5 pts}\\
En tapant la commande {\tt tree} dans son répertoire personnel (c'est à dire dans {\tt /home/alfred/}), un utilisateur d'un système Linux a obtenu le résultat ci-contre. \\
\tei{arbo.eps}{0.25}{2}{On suppose que le répertoire actuel est {\tt /home/alfred/}.
\QListe
\item Ecrire une commande permettant de se déplacer dans le dossier {\tt Factures} en donnant un chemin relatif\\
\lpo[1]
\item Ecrire une commande permettant d'y creer les dossiers {\tt Eau} et {\tt Electricité}\\
\lpo[1]
\item Donner une commande permettant de se rendre dans le dossier {\tt Images} en donnant un chemin absolu.\\
\aide\; on rappelle que l'arborescence ci-dessus est celle de {\tt /home/alfred/}\\
\lpo[1]
\FinListe
\ \vspace{0.1cm}
}
\QListe[1]
\item Quelle commande permet de lister le contenu d'un dossier ? \\
\lpo[1]
\item Dans chaque cas écrire une commande permettant de modifier les droits sur le fichier \texttt{monfichier} de la façon indiquée :
\SQListe
\item Supprimer le droit de lecture pour le groupe et pour les autres\\
\lpo[1]
\item Ajouter le droit d'écriture et de lecture pour le propriétaire et le groupe\\
\lpo[1]
\FinListe
\FinListe
\vspace{0.3cm}

\Exo{Un peu de pratique !}{5 pt}
\begin{tcolorbox}[title=\textcolor{black}{\danger \; Attention !},colbacktitle=lightgray]
     Pour cet exercice, la ligne de commande et \textbf{uniquement} la ligne de commande est utilisée, aussi la connection au système se fera \textbf{sans interface graphique}. Toute connection en mode graphique entraîne la nullité de la totalité des réponses fournies !
\end{tcolorbox}
\QListe
    \item Se connecter et créer le fichier réponse
\SQListe
\item A l'écran graphique de connection, appuyer simultanément sur \keys{\ctrl + \Alt + F4}, puis suivre les instructions à l'écran pour vous connecter (entrer simplement votre \textit{login} c'est à dire votre identifiant de connection puis votre mot de passe)
\item Se rendre dans le dossier {\tt Evaluations}, y créer le dossier {\tt DS1}, se déplacer dans {\tt DS1} et y créer le fichier {\tt ex2.txt}.
\item Ecrire dans le fichier {\tt ex2.txt} la phrase suivante : {\tt Etape 1 réussie !} \\
    \aide \; On rapelle que la commande {\tt nano} permet d'éditer un fichier dans le terminal
\FinListe
    \item Heure de connection
\SQListe
\item Lire l'aide de la commande  {\tt last}
\item Dans votre fichier réponse écrire {\tt Etape 2} suivi de l'heure (format {\tt hh:mm}) de votre toute dernière connection au système (en expliquant comment vous avez obtenu cette information).
\FinListe
\item Un peu de Python
\SQListe
    \item En utilisant Python, calculer : $ 2022^{3}-2^{32}$
    \item Importer la valeur de {\tt pi} depuis le module {\tt math}, quelle est sa valeur ?
    \item Quitter l'interpréteur Python et écrire dans votre fichier {\tt Etape 3} suivi des réponses au deux questions précédentes.
\FinListe

\FinListe
\begin{tcolorbox}[title=\textcolor{black}{\faHandPointRight \; Remarques},colbacktitle=lightgray]
    \begin{itemize}
        \item A la fin de l'évaluation : utiliser {\tt exit} ou {\tt logout} pour vous déconnecter.
        \item Pour afficher l'interface graphique de connection habituelle appuyer simultanément sur \keys{\ctrl + \Alt + F1}.
    \end{itemize}
\end{tcolorbox}
\end{document}

