\documentclass[12pt,a4paper]{article}

\usepackage{Act}
\usepackage{listings}

\begin{document}
\input{\detokenize{/home/fenarius/Travail/Cours/Commun/latex/Macros.tex}}

\DevoirNSI{Entrainement à l'épreuve pratique}{\Term}\vspace{0.2cm}

\Exo{Nombre d'occurences}{} \\
Ecrire une fonction {\tt nb\_occurences} qui prend en argument un entier {\tt n} et une liste d'entiers {\tt liste} et qui renvoie le nombre d'apparitions de {\tt n} dans {\tt liste}.\\
Par exemple, {\tt nb\_occurences(5,[12,6,5,18,11,11,5,6,5])} renvoie 3 puisque l'entier 5 apparaît trois fois dans la liste {\tt [12,6,5,18,11,11,5,6,5]}
\vspace{0.3cm}


\textbf{Exemples :} 
\begin{lstlisting}
>>> nb_occurences(7,[1,7,16,9,6])
1
>>> nb_occurences(11,[45,6,18,9])
0
>>> nb_occurences(0,[1,0,1,0,0,0,1,0,1])
5
\end{lstlisting}

\end{document}