\documentclass[12pt,a4paper]{article}

\usepackage{Act}
\usepackage{listings}

\begin{document}
\input{\detokenize{/home/fenarius/Travail/Cours/Commun/latex/Macros.tex}}
\pythonmode

\DevoirNSI{Entrainement à l'épreuve pratique}{\Term}\vspace{0.2cm}

\Exo{Appartient}{} 

Ecrire une fonction {\tt appartient} qui prend en paramètre un entier {\tt n} et une liste d'entiers {\tt entiers} et renvoie {\tt True} si {\tt n} apparaît dans {\tt entiers} et {\tt False} sinon.
\begin{lstlisting}
>>> appartient(10,[4,17,11])
False
>>> appartient(10,[])
False
>>> appartient(2,[8,5,1,1,1,2,7])
True
>>> appartient(0,[0,1,1,0,0,0,1])
True
>>> appartient(2,[0,1,1,0,0,0,1])
False
\end{lstlisting}

\vspace{0.2cm}

\Exo{Palindrome}{}

Une chaine de caractères est un \textit{palindrome} lorsqu'elle peut être lue indifféremment de gauche à droite ou de droite à gauche, par exemple \og KAYAK \fg ou \og ANNA \fg sont des palindromes.
Ecrire une fonction {\tt est\_palindrome} qui prend en argument une chaîne de caractères et renvoie {\tt True} si cette chaîne est un palindrome et {\tt False} sinon. 

\textbf{Exemples :} 
\begin{lstlisting}
>>> est_palindrome("toto")
False
>>> est_palindrome("ressasser")
True
>>> est_palindrome("")
True
>>> est_palindrome("radar")
True
>>> est_palindrome("p")
True
>>> est_palindrome("retirer")
False
\end{lstlisting}

\vspace{0.2cm}

\Exo{Bonus}{}
\QListe
\item Si vous avez proposé une solution récursive à l'exercice 2 alors donner une solution itérative et inversement.
\item Pour l'exercice 1, lorsqu'un la liste {\tt entiers} est triée, un algorithme plus rapide existe pour déterminer si un entier {\tt n} y figure ou non.
    \SQListe
    \item Rappeler le nom de cet algorithme et expliquer rapidement son fonctionnement.
    \item Expliquer pourquoi cet algorithme est \textit{plus rapide}.\\
        \aide \; On pourra indiquer simplement leur complexité en temps.
    \item Ecrire une implémentation en Python de cet algorithme 
    \FinListe
\FinListe
\end{document}