\documentclass[12pt,a4paper]{article}

\usepackage{Act}
\usepackage{listings}

\begin{document}
\input{\detokenize{/home/fenarius/Travail/Cours/Commun/latex/Macros.tex}}

\EP{6}{2022--2023}
\pythonmode

\Exo{Somme des éléments positifs d'une liste}{}

Ecrire une fonction {\tt somme\_positif} qui prend en argument une liste d'entiers {\tt entiers} et renvoie la somme des entiers positifs de cette liste.


\textbf{Exemples :}
\begin{lstlisting}
>>> somme_positif([-1,-6,-8,3])
3
>>> somme_positif([5,-12,0,1,-4,7])
13
>>> somme_positif([])
0
>>> somme_positifs([-7,-11,-2])
0
\end{lstlisting}

\vspace{0.2cm}

\Exo{Unicité des éléments d'une liste}{}

On veut écrire une fonction {\tt unique} qui prend en argument une liste  {\tt liste}  et renvoie {\tt True} si cette liste ne contient pas d’éléments en double et {\tt False sinon}. Pour cela on propose l'algorithme suivant :
\begin{itemize}
    \item On crée une liste {\tt deja\_vu} initialement vide
    \item On parcourt chaque élément de la liste {\tt liste}, si un élément ne se trouve pas dans {\tt deja\_vu} alors on le rajoute sinon c'est qu'on l'a déjà rencontré et on renvoie {\tt False}
    \item A la fin du parcours on renvoie {\tt False}
\end{itemize} 

\begin{lstlisting}
def unique(liste):
"""Renvoie True si les élements de liste sont uniques, False sinon"""
deja_vu = ...
for ..... in ....:
    if ..... in deja_vu:
        return .....
    else:
        ...........
return ....
\end{lstlisting}

\textbf{Exemples}
\begin{lstlisting}
>>> unique([5,7,8,5,12])
False
>>> unique([5,7,8,12])
True
>>> unique([])
True
>>> unique([2])
True
>>> unique([2,0,1,8,9,2])
False
\end{lstlisting}
\end{document}
