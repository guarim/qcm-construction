\documentclass[12pt,a4paper]{article}

\usepackage{Act}
\usepackage{listings}

\begin{document}
\input{\detokenize{/home/fenarius/Travail/Cours/Commun/latex/Macros.tex}}

\EP{11}{2022--2023}
\pythonmode

\Exo{Nom de variables en python}{}

En Python, un nom de variable valide, :
\begin{itemize}
    \item doit commencer par une lettre ou un \textit{underscore} : {\tt \_}
    \item ne contient pas d'espace
    \item peut contenir des lettres ou des chiffres
    \item ne doit pas être un des mots réservés du langage. Les mots réservés sont : {\tt and, as, assert, break, class, continue, def, del, elif, else, except, exec, finally, for, from, global, if, import, in, is, lambda, not, or, pass, print, raise, return, try, while, with, yield.}
\end{itemize}
Ecrire une fonction {\tt nom\_valide} qui prend en argument une chaine de caractères et renvoie {\tt True} si cette chaine de caractère représente un nom de variable valide et {\tt False} sinon.

\aide \; Aide \\    
    On pourra créer une liste contenant les mots réservés et utiliser {\tt in} pour savoir si une chaine de caractères se trouve ou pas dans cette liste.

\textbf{Exemples :}
\begin{lstlisting}
>>> nom_valide("ma_variable")
True
>>> nom_valide("la temperature")
False
>>> nom_valide("5cases")
False
>>> nom_valide("for")
False
\end{lstlisting}

\vspace{0.2cm}

\Exo{Insertion dans une liste déjà triée}{}

Pour insérer un élément dans une liste \textit{déjà triée par ordre croissant}, on propose l'algorithme suivant :
\begin{enumerate}
	\item Ajouter l'élément à la fin de la liste
	\item Echanger l'élément avec son voisin de gauche tant qu'il est plus petit et que le début de liste n'est pas atteint
\end{enumerate}
Compléter l'implémentation suivante de cette algorithme
\begin{lstlisting}
def insertion(element,liste):
    '''Insère element dans  liste (qui est déjà triée)'''
    liste........(element)
    pos = ......
    while pos>... and liste[pos]<.....:
        liste[pos],liste[pos-1] =  ........... , ............ 
        pos = ......
\end{lstlisting}

\textbf{Exemple}
\begin{lstlisting}
    >>> liste = [1,4,7,13,17]
    >>> insertion(8,liste)
    >>> liste
    [1,4,7,8,13,17]
\end{lstlisting}

\end{document}
