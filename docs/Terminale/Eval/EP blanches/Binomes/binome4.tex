\documentclass[11pt,a4paper]{article}

\usepackage{Act}
\usepackage{listings}

\begin{document}
\input{\detokenize{/home/fenarius/Travail/Cours/Commun/latex/Macros.tex}}

% niveau et nom du projet à définir ici
\newcommand{\niveau}{\Term}

\fancyhead[L]{\niveau} \fancyhead[R]{Lycée Hintermann-Afféjee}
\fancyfoot[C]{}
\begin{center}
\framebox{\makebox[0.986\textwidth][c]{\Large \bf Entrainement à l'épreuve pratique}} \\
\end{center}

\begin{tabularx}{0.47\textwidth}{|p{1.7cm}X|}
    \hline
    \multicolumn{2}{|c|}{\textbf{Elève A}} \\
    Nom    & \dotfill \\
    Prénom & \dotfill \\
    \hline
\end{tabularx} \hfill
\begin{tabularx}{0.47\textwidth}{|p{1.7cm}X|}
    \hline
    \multicolumn{2}{|c|}{\textbf{Elève B}} \\
    Nom    & \dotfill \\
    Prénom & \dotfill \\
    \hline
\end{tabularx} 

\vspace{0.25cm}
\textbf{Enoncé}

Ecrire une fonction {\tt maximum} qui prend en argument une liste d'entiers \textit{non vide} {\tt entiers} et renvoie le maximum des éléments de cette liste. 


\textbf{Exemples :}
\begin{lstlisting}
>>> maximum([12, 11, 8, 20, 15])
20
>>> maximum([-5, -12, -11, -4, -7])
-4
\end{lstlisting}

\vspace{0.1cm}
\renewcommand{\arraystretch}{1.2}
\begin{tabularx}{\textwidth}{|X|}
    \hline
    \ding{43} {\bf Etape \ding{182} : elève A/B} : Comprendre l'énoncé, proposer des tests pour la fonction demandée \\
    \ \dotfill \\
    \ \dotfill \\
    \ \dotfill \\
    \ \dotfill \\
    \ \dotfill \\
    \ \dotfill \\
    \hline
\end{tabularx}

\vspace{0.1cm}
\renewcommand{\arraystretch}{1.2}
\begin{tabularx}{\textwidth}{|X|}
    \hline
    \ding{43} {\bf Etape \ding{183} : élève A} : Ecrire en français un algorithme pour résoudre le problème \\
    \ \dotfill \\
    \ \dotfill \\
    \ \dotfill \\
    \ \dotfill \\
    \ \dotfill \\
    \ \dotfill \\
    \ \dotfill \\
    \ \dotfill \\
    \hline
\end{tabularx}

\vspace{0.1cm}
\renewcommand{\arraystretch}{1.2}
\begin{tabularx}{\textwidth}{|X|}
    \hline
    \ding{43} {\bf Etape \ding{184} : élève B} : Traduire cet algorithme en Python et le tester \\
    \ \dotfill \\
    \ \dotfill \\
    \ \dotfill \\
    \ \dotfill \\
    \ \dotfill \\
    \ \dotfill \\
    \ \dotfill \\
    \ \dotfill \\
    \hline
\end{tabularx}

\end{document}