\documentclass[12pt,a4paper]{article}

\usepackage{Act}
\usepackage{listings}

\begin{document}
\input{\detokenize{/home/fenarius/Travail/Cours/Commun/latex/Macros.tex}}

\EP{9}{2022--2023}
\pythonmode

\Exo{Nombre d'occurrence}{}

Écrire une fonction  {\tt nombre\_occurrences} qui prend en paramètre un entier {\tt n} et une liste d'entiers {\tt entiers} et qui renvoie le nombre d’occurrences de {\tt n} dans {\tt entiers}, c'est à dire le nombre de fois où l'entier {\tt n} apparaît dans la liste {\tt entiers}:

\textbf{Exemples :}
\begin{lstlisting}
>>> nombre_occurrences(5,[1,4,5,-2,-5,12,5,8])
2
>>> nombre_occurrences(0,[1,0,1,0,0,0,1,0])
5
>>> nombre_occurrences(7,[1,2,3,4,5,6]) 
0
\end{lstlisting}

\vspace{0.2cm}

\Exo{Maison en  POO}{}

Dans une agence immobilière, on modélise les maisons à vendre par :
\begin{itemize}
    \item[\textbullet] une surface totale
    \item[\textbullet] un nombre de pièces
    \item[\textbullet] un quartier 
    \item[\textbullet] un prix 
\end{itemize}

\QListe
\item Compléter la classe {\tt Maison} ci-dessous représentant cette modélisation :

\begin{lstlisting}
    class Maison:
        def __init__(self,surface,nb_pieces,quartier,prix)
            self.surface = .....
            self.nb_pieces = .....
            self.quartier = .....
            self.prix = .....
\end{lstlisting} 

\item Créer un objet de classe {\tt Maison}, nommé {\tt ex\_maison} et qui représente une maison de 150 m\textsuperscript{2} ayant 4 pièces situé dans le quartier du Moufia et valant \numprint{330000} euros.

\item Compléter cette classe en écrivant une méthode {\tt modifie\_prix} qui prend en argument un {\tt nouveau\_prix} et affecte ce nouveaux prix à la maison.


\FinListe



\end{document}
