\documentclass[12pt,a4paper]{article}

\usepackage{Act}
\usepackage{listings}

\begin{document}
\input{\detokenize{/home/fenarius/Travail/Cours/Commun/latex/Macros.tex}}

\EP{5}{2022--2023}
\pythonmode

\Exo{Recherche simple}{}

Ecrire une fonction {\tt est\_dans} qui prend en argument un entier {\tt n} et une liste d'entiers {\tt entiers} et qui renvoie {\tt True} si {\tt n} est dans la liste {\tt entiers} et {\tt False} sinon.

\textbf{Exemples :}
\begin{lstlisting}
>>> est_dans(4,[1,6,8,3])
False
>>> est_dans(4,[5,12,0,1,4,7])
True
>>> est_dans(4,[])
False
>>> est_dans(7,[7])
True
\end{lstlisting}

\vspace{0.2cm}

\Exo{Fusion de deux listes déjà triées}{}

On veut écrire une fonction {\tt fusion} qui prend en argument deux listes \textit{déjà triées}  et renvoie ces deux listes fusionnées. Pour cela, on propose l'algorithme récursif suivant :
\begin{itemize}
    \item si l'une des deux listes est vide, alors on renvoie l'autre. Par exemple {\tt fusion([7,9],[])} renvoie la première liste c'est à dire {\tt [7,9]}.
    \item sinon, on renvoie le minimum entre les deux premiers éléments de chacune des deux listes suivie de la fusion du reste. Par exemple {\tt fusion([7,9],[8,10])} doit renvoyer \\ {\tt [7] + fusion([9],[8,10])}
\end{itemize} 

\aide \; Rappel \\
Si {\tt l} est une liste de Python alors {\tt l[1:]} est une copie de cette liste à partir de son deuxième élément.

\begin{lstlisting}
    def fusion(l1,l2):
    """fusion récursive des deux listes l1 et l2 déjà triées"""
    if l1 == []:
        ........
    if l2 .....:
        ..........
    if .....<.....:
        return ..........
    else:
        return ..........
\end{lstlisting}

\textbf{Exemples}
\begin{lstlisting}
>>> fusion([1,5,7],[2,6,18,20])
[1, 2, 5, 6, 7, 18, 20]
>>> fusion([5,18],[])
[5, 18]
>>> fusion([],[2,5,9])
[2, 5, 9]
>>> fusion([5,8],[1,2,3,4,6,7,9,10])
[1, 2, 3, 4, 5, 6, 7, 8, 9, 10]
\end{lstlisting}
\end{document}
