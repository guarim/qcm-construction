\documentclass[12pt,a4paper]{article}

\usepackage{Act}
\usepackage{listings}

\begin{document}
\input{\detokenize{/home/fenarius/Travail/Cours/Commun/latex/Macros.tex}}

\EP{10}{2022--2023}
\pythonmode

\Exo{Mot équilibré}{}

On dit qu'un mot est \textit{équilibré} lorsqu'il contient autant de voyelles que de consonnes. Par exemple le mot {\tt rapide} est un mot équilibré, par contre {\tt lent} n'est pas équilibré. Ecrire  une fonction {\tt est\_equilibre} qui prend en paramètre un mot {\tt mot} et renvoie {\tt True} si {\tt mot} est équilibré et {\tt False} sinon.
On suppose que {\tt mot} est constitué uniquement de lettres minuscules et non accentuées et on rappelle que les voyelles sont les lettres {\tt a,e,i,o,u} et {\tt y}.

\textbf{Exemples :}
\begin{lstlisting}
>>> est_equilibre("oiseau")
False
>>> est_equilibre("patate")
True
>>> est_equilibre("toux")
True
\end{lstlisting}

\vspace{0.2cm}

\Exo{Conversion en binaire}{}

Pour rappel, la conversion d'un nombre entier en binaire peut s'effectuer à l'aide de \textit{l'algorithme des divisions successives} qui consiste à :
\begin{itemize}
\item[\textbullet] effectuer la division euclidienne de $n$ par 2, soit $q$ le quotient et $r$ le reste.
\item[\textbullet] si le quotient est 0, on s'arrête l'écriture binaire est la suite des restes prises dans l'ordre inverse. Sinon on recommence l'étape 1 en remplaçant $n$  par $q$
\end{itemize}

Par exemple pour $n=77$ :
\begin{itemize}
 \item[\textbullet] $77 = 2 \times 38$ + 1, comme $38 \neq 0$ on continue
 \item[\textbullet] $38 = 2 \times 19$ + 0, comme $19 \neq 0$ on continue
 \item[\textbullet] $19 = 2 \times 9$ + 1
 \item[\textbullet] $9 = 2 \times 4$ + 1
 \item[\textbullet] $4 = 2 \times 2$ + 0 
 \item[\textbullet] $2 = 2 \times 1$ + 0
 \item[\textbullet] $1 = 2 \times 0$ + 1 arrêt car le quotient est 0.
\end{itemize}
L'écriture binaire de 77 est la suite des restes prises dans l'ordre inverse : $1001101$.


Compléter la fonction ci-dessous qui implémente cet algorithme, on rappelle que le quotient dans la division euclidienne de $n$ par 2 s'obtient avec {\tt n//2} et le reste avec {\tt n\%2} :
\begin{lstlisting}
def binaire(n):
    # On effectue la première division euclidienne, le reste doit être converti en chaine de caractères pour initialiser l'écriture binaire
    ecriture_binaire = str(...)
    n = n .....
    # Tant que le quotient n'est pas nul, on recommence
    while  ..... :
        ecriture_binaire = str(...) + ....
        n = .....
    return ecriture_binaire
\end{lstlisting}

\textbf{Exemples : }
\begin{lstlisting}
>>> binaire(77)
'1001101'
\end{lstlisting} 

\end{document}
