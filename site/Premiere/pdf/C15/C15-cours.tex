\PassOptionsToPackage{dvipsnames,table}{xcolor}
\documentclass[10pt]{beamer}
\usepackage{Cours}

\begin{document}

\input{\detokenize{/home/fenarius/Travail/Cours/fabricenativel.github.io/latex//MacrosCours.tex}}
\setcounter{numchap}{15}

\pythonmode

\newcommand{\IHM}{\cnum Interface Homme-Machine}

\pythonmode

% Capteurs et actionneurs
\begin{frame}
    \mframe{\IHM}
    \begin{block}{Définitions}
        Dans un système informatique embarqué (par exemple un objet connecté) :
        \begin{itemize}
            \item<2-> les \textcolor{red}{capteurs} récupèrent des informations du monde réel (pression, température, ...), ce sont des périphériques d'entrée.
            \item<3-> le \textcolor{red}{processeur} traite ces informations en exécutant un programme,
            \item<4-> les \textcolor{red}{actionneurs} agissent dans le monde réel (affichage, son, moteurs,...), ce sont des périphériques de sortie.
        \end{itemize}
    \end{block}
\end{frame}

\begin{frame}
    \mframe{\IHM}
    \begin{block}{IHM}
        Une interface Homme-Machine \textcolor{red}{\sc ihm} l'ensemble des méthodes qui permettent une communication bidirectionnelle entre l'utilisateur et le système informatique embarqué.
        La mise au point d'une IHM, répond normalement à un cahier des charges strict qui indique toutes les fonctionnalités attendues.
    \end{block}
    \begin{exampleblock}{Carte Micro:bit}
        \onslide<2->La mise en oeuvre des concepts du cours a été faite en utilisant une carte Micro:Bit :
        \begin{itemize}
            \item<3-> La carte possède des capteurs (boutons, accéléromètre, ...),
            \item<4-> et des actionneurs (matrice de 5x5 leds, broche {\sc gpio}).
            \item<5-> Elle est équipée d'un processeur pour traiter les données.
        \end{itemize}
    \end{exampleblock}
\end{frame}

\end{document}



